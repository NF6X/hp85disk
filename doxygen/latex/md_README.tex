\hypertarget{md_README_README}{}\section{R\+E\+A\+D\+ME}\label{md_README_README}
\section*{Documentation}


\begin{DoxyItemize}
\item Please use this link for the Documentation and this R\+E\+A\+D\+ME
\begin{DoxyItemize}
\item \href{https://rawgit.com/magore/hp85disk/V2/doxygen/html/index.html}{\tt https\+://rawgit.\+com/magore/hp85disk/\+V2/doxygen/html/index.\+html}
\item \href{doxygen/html/index.html}{\tt index}
\end{DoxyItemize}
\end{DoxyItemize}

\subsection*{H\+P85 Disk Emulator Copyright (C) 2014-\/2020 Mike Gore}


\begin{DoxyItemize}
\item New Board layout design by Jay Hamlin (C) 2018-\/2020
\item See \hyperlink{COPYRIGHT_8md}{C\+O\+P\+Y\+R\+I\+G\+HT} for a full copyright notice for the project
\end{DoxyItemize}

\subsection*{H\+P85 disk emulator V2 circuit board layout design by (C) 2018-\/2020 Jay Hamlin}

\subsection*{V2 board design -\/ github V2 branch targets the new board by Jay Hamlin}

\subsection*{V2 code is now working}


\begin{DoxyItemize}
\item \href{board/V2/releases}{\tt Jay Hamlin designed this board}
\item V2 hardware adds
\begin{DoxyItemize}
\item G\+P\+IB B\+US drivers
\begin{DoxyItemize}
\item 48\+Mma drive required by the G\+P\+IB spec
\end{DoxyItemize}
\item I2C level conveters and standard Qwiic Bus interface -\/ 3.\+3V
\begin{DoxyItemize}
\item optional R\+TC chips like the D\+S3231
\item L\+CD displays -\/ work in progress
\end{DoxyItemize}
\item Advanced Hardware Reset circuit
\item Full size SD card interface with Card detect
\end{DoxyItemize}
\end{DoxyItemize}

\subsection*{H\+P85 disk emulator V1 board design (C) 2014-\/2020 Mike Gore}

\subsection*{board/\+V1/\+R\+E\+A\+D\+M\+E.\+md \char`\"{}\+V1 board readme\char`\"{}}


\begin{DoxyItemize}
\item Limited control and B\+US drive power
\begin{DoxyItemize}
\item About half of the 48\+Mma drive required by the G\+P\+IB spec
\item However we can read any pin any time -\/ useful for tracing/debugging
\end{DoxyItemize}
\item R\+TC D\+S1307 for time stamping
\item My original board design without G\+P\+IB buffers
\end{DoxyItemize}

\subsection*{H\+P85 emulator board design Makefile configuration options V1 and V2 boards}

\subsection*{V2 design enables control of G\+P\+IB B\+US drivers (drivers do not exist on V1 hardware)}

\subsection*{V1 Board design Makefile configuration (github main branch)}


\begin{DoxyItemize}
\item B\+O\+A\+RD=1 (2 also works V2 because the G\+P\+IB drivers are not connected)
\item P\+P\+R\+\_\+\+R\+E\+V\+E\+R\+S\+E\+\_\+\+B\+I\+TS=0 (P\+PR bits reverse in hardware) \subsection*{V2 Board design Makefile configuration (github V2 branch)}
\end{DoxyItemize}


\begin{DoxyItemize}
\item B\+O\+A\+RD=2
\item P\+P\+R\+\_\+\+R\+E\+V\+E\+R\+S\+E\+\_\+\+B\+I\+TS=1 (P\+PR bits reversed in software) 


\end{DoxyItemize}

\subsection*{Features}


\begin{DoxyItemize}
\item This project emulates G\+P\+IB drives and H\+P\+GL printer for the H\+P85A and H\+P85B computers.
\begin{DoxyItemize}
\item A\+M\+I\+GO drives work with H\+P85A
\item S\+S80 drives work with H\+P85B (or H\+P85A with prm-\/85 wth modified E\+MS and Electronic disk rom add on board see links)
\begin{DoxyItemize}
\item You may have up to 4 disks with V1 hardware and 8 with V2 hardware
\end{DoxyItemize}
\item Printer emulator -\/ can capture and save printer data to a time stamped file.
\item There is a U\+SB serial interface that can take many commands -\/ type help
\begin{DoxyItemize}
\item You can use a serial terminal program to acces it -\/ See Makefile for Baud and terminal command
\item There are many commands that you can use, type \char`\"{}help for a list\char`\"{}
\item Any key press halts the emulator and waits for a user command.
\begin{DoxyItemize}
\item After finishing any user commend it returns to G\+P\+IB disk emulation.
\end{DoxyItemize}
\end{DoxyItemize}
\item Each emulated disk image is a file on a F\+A\+T32 formatted S\+D\+C\+A\+RD.
\item \href{sdcard/hpdisk.cfg}{\tt hpdisk.\+cfg} fully defines each disk image on SD Card
\begin{DoxyItemize}
\item Disk images are L\+IF encoded files that are compatible with H\+P85\+A/B and many other computers
\item Missing disk image files are created automatically if
\end{DoxyItemize}
\item L\+IF manipulation tools are built into the emulator -\/ see next sections
\item R\+TC can be used for time stamping plot files and files added into lif images
\end{DoxyItemize}
\end{DoxyItemize}

\subsection*{L\+IF tools are built into emulator firmware}

\subsection*{L\+IF tools are also created for your operating system}


\begin{DoxyItemize}
\item Type \char`\"{}lif help\char`\"{} in the emulator for a full list of commands
\begin{DoxyItemize}
\item See the top of \href{lif/lifutils.c}{\tt lifutils.\+c} for full documentation and examples.
\end{DoxyItemize}
\item The emulator will automatically create missing L\+IF images defined in hpdisk.\+cfg on the S\+D\+C\+A\+RD
\end{DoxyItemize}

For the specific L\+IF E010..E013(hex) type records tools can translate to and from plain A\+S\+C\+II files.
\begin{DoxyItemize}
\item add ascii file to L\+IF image
\item extract ascii files from L\+IF image
\item Add correct timestamp if you have an R\+TC
\end{DoxyItemize}

add binary file to L\+IF image
\begin{DoxyItemize}
\item extract binary file from L\+IF image
\begin{DoxyItemize}
\item Extracted images have a 256 byte volume header, 256 byte directory followed by a file.
\end{DoxyItemize}
\item delete file in L\+IF image
\item rename file in L\+IF image
\item list directory
\begin{DoxyItemize}
\item Display time stamps if they were set
\end{DoxyItemize}
\item create L\+IF image with options
\item Can automaticcaly create missing images if defined in hpdisk.\+cfg
\begin{DoxyItemize}
\item Type \char`\"{}lif help\char`\"{} in the emulator for a full list of commands
\item See the top of \href{lif/lifutils.c}{\tt lifutils.\+c} for full documentation and examples.
\end{DoxyItemize}
\end{DoxyItemize}

\subsection*{Tele\+Disk to L\+IF extractor tool (updated) included -\/ see  \hyperlink{lif_2README_8md}{lif/\+R\+E\+A\+D\+M\+E.\+md} \char`\"{}\+L\+I\+F R\+E\+A\+D\+M\+E.\+md\char`\"{}}


\begin{DoxyItemize}
\item \href{lif/t202lif}{\tt td02lif} \href{lif/85-SS80.TD0}{\tt 85-\/\+S\+S80.\+T\+D0} \href{lif/85-SS80.LIF}{\tt 85-\/\+S\+S80.\+L\+IF}
\end{DoxyItemize}





\subsection*{SD Card requirments}


\begin{DoxyItemize}
\item The H\+P85 is sensitive to write delays so we need SD cards with fast random writes.
\item I have found that the Sandisk Extreme and Sandisk Extreme Pro cards work best.
\item Why ? Each block that is written reads, erases then rewrites a new internal flash page (these can be over a megabyte and so take time). Most SD cards are optimized for sequencial writting and do not do well with random writes. There is a huge difference in various cards on the market. Look for the cards with the best 4K random write times. Some SD cards are so slow it will cause the H\+P85 to timeout waiting for the card. Best source of benchmark information is looking for recent Raspberry Pi SD card benchmarks -\/ specifically 4k random write -\/ faster is better.
\end{DoxyItemize}





\subsection*{OS Requirements for software building}


\begin{DoxyItemize}
\item I used Ubuntu 18.\+04,16.\+04\+L\+TS and 14.\+04\+L\+TS when developing the code
\begin{DoxyItemize}
\item It should be easy to setup the same build with Windows gcc tools.
\end{DoxyItemize}
\end{DoxyItemize}

\subsection*{Install Ubuntu Packages required for Building}


\begin{DoxyItemize}
\item sudo bash
\item {\itshape apt-\/get update}
\item {\itshape apt-\/get install aptitude make build-\/essential binutils gcc}
\item {\itshape aptitude --with-\/recommends install minicom avr-\/libc avra avrdude avrdude-\/doc avrp binutils-\/avr gcc-\/avr gdb-\/avr}
\item pip install pyserial
\end{DoxyItemize}

\subsection*{Compiling A\+VR code and stand alone L\+IF tools}


\begin{DoxyItemize}
\item Please update B\+A\+UD, P\+O\+RT, B\+O\+A\+RD, P\+P\+R\+\_\+\+R\+E\+V\+E\+R\+S\+E\+\_\+\+B\+I\+TS and R\+T\+C\+\_\+\+S\+U\+P\+P\+O\+RT for your platform
\begin{DoxyItemize}
\item B\+O\+A\+RD is the version of the hardware
\begin{DoxyItemize}
\item 1 = original github design
\begin{DoxyItemize}
\item also set P\+P\+R\+\_\+\+R\+E\+V\+E\+R\+S\+E\+\_\+\+B\+I\+TS=0
\end{DoxyItemize}
\item 2 = new hardware design with G\+P\+IB bus buffers
\end{DoxyItemize}
\item P\+P\+R\+\_\+\+R\+E\+V\+E\+R\+S\+E\+\_\+\+B\+I\+TS
\begin{DoxyItemize}
\item 0 = original github design
\item 1 = cirrently needed for V2 hardware in development
\end{DoxyItemize}
\item R\+T\+C\+\_\+\+S\+U\+P\+P\+O\+RT
\begin{DoxyItemize}
\item You have a D\+S1307 command compatible R\+TC chip -\/ the D\+S3231 is the 3.\+3V version
\begin{DoxyItemize}
\item This will time stamp plot files and add time stamps inside lif images
\end{DoxyItemize}
\end{DoxyItemize}
\item B\+A\+UD is the serial B\+A\+UD rate used to communicate with the emulator interface
\begin{DoxyItemize}
\item Note\+: I am using 500000 baud as default -\/ your OS may not support that -\/ linux does
\item Try 115200 if it does not
\end{DoxyItemize}
\item P\+O\+RT is the serial P\+O\+RT device name used to communicate with the emulator interface
\end{DoxyItemize}
\item {\itshape make clean}
\item {\itshape make}
\end{DoxyItemize}

\subsection*{Flashing the firmware to the A\+VR with avrdude and programmer}


\begin{DoxyItemize}
\item Note\+: J\+T\+AG is disabled so we can use port C bits to control the G\+P\+IB drivers
\item You will need and A\+VR programmer supported by avrdude (part of avrtools)
\begin{DoxyItemize}
\item I am using atmelice\+\_\+isp but the \mbox{[}Makefile\mbox{]}(Makefile) as example for\+:
\item {\itshape avrispmk\+II atmelice atmelice\+\_\+dw atmelice\+\_\+isp atmelice\+\_\+pdi}
\item If you wish to another programmer then update the \char`\"{}flash\char`\"{} avrdude command line in the \mbox{[}Makefile\mbox{]}(Makefile).
\item There is an example with the A\+VR mkii programmer as well.
\end{DoxyItemize}
\item {\itshape make flash}
\begin{DoxyItemize}
\item This will use {\itshape avrdude} to flash the firfmware
\item N\+O\+TE\+: When finished {\itshape make} will call call shell script to launch a terminal program for debugging
\begin{DoxyItemize}
\item These scripts are called {\itshape miniterm} or {\itshape term} in the project folder
\item They expect a baud rate and device name as an option. See\+: \mbox{[}Makefile\mbox{]}(Makefile)
\end{DoxyItemize}
\end{DoxyItemize}
\end{DoxyItemize}

\subsection*{Building Doxygen documentation for the project -\/ optional}


\begin{DoxyItemize}
\item {\itshape aptitude install --with-\/recommends doxygen doxygen-\/doc doxygen-\/gui doxygen-\/latex}
\item {\itshape If you omit this you will have to update the \mbox{[}Makefile\mbox{]}(Makefile) to omit the steps}
\end{DoxyItemize}





\subsection*{Using the emulator with examples}


\begin{DoxyItemize}
\item See \href{sdcard/sdcard.cfg}{\tt sdcard.\+cfg} for configuration settings and setting and documentation.
\begin{DoxyItemize}
\item Printer capture is configured currently for my H\+P54645D scope
\begin{DoxyItemize}
\item The following example works for an H\+P85 attached to the emulator via G\+P\+IB bus.
\begin{DoxyItemize}
\item P\+R\+I\+N\+T\+ER IS 705
\item P\+L\+I\+ST
\end{DoxyItemize}
\end{DoxyItemize}
\item Disk images in \mbox{[}sdcard\mbox{]}(sdcard) folder drive and configuration settings
\begin{DoxyItemize}
\item First Amigo 9121D disk at 710 for my H\+P85B (with 85B R\+O\+Ms)
\item Second Amigo 9121D disk at 710 for my H\+P85B (with 85B R\+O\+Ms)
\item First S\+S80 H\+P9134L disk at 720 for my H\+P85A (with 85A R\+O\+Ms)
\item Second S\+S80 H\+P9134L disk at 730 for my H\+P85A (with 85A R\+O\+Ms)
\end{DoxyItemize}
\item How to use the examples with your H\+P85
\begin{DoxyItemize}
\item Copy the files inside the project \mbox{[}sdcard\mbox{]}(sdcard) folder to the home folder of a fat32 formatted drive
\begin{DoxyItemize}
\item All image files and configuration must be in the home folder only -\/ not in a subdirectory.
\item You may store other user files in sub folders of your choosing.
\end{DoxyItemize}
\item Verify \href{sdcard/hpdisk.cfg}{\tt hpdisk.\+cfg} configuration settings for your computer
\item Insert card into emulator
\item Attract G\+P\+IB cables
\item Power on emulator
\item Power on your computer last!
\begin{DoxyItemize}
\item The emulator M\+U\+ST be running and attached to your computer first!
\item The H\+P85 O\+N\+LY checks for drives at power up. 

 \subsection*{Understanding Drive G\+P\+IB B\+US addressing and Parallel Poll Response (P\+PR) -\/ H\+P85A vs. H\+P85B}
\end{DoxyItemize}
\end{DoxyItemize}
\end{DoxyItemize}
\end{DoxyItemize}

While G\+P\+IB devices can have address between 0 and 31 you can have no more than 8 disk drives.
\begin{DoxyItemize}
\item A\+LL disk drives are required to respond to a P\+PR query by the (H\+P85) controller.
\begin{DoxyItemize}
\item P\+PR query is done when the controller in charge (H\+P85) pulls A\+TN and E\+OI low.
\item P\+PR response occurs when a disk drive pulls one G\+P\+IB bus data line low in response.
\begin{DoxyItemize}
\item You can only have 8 of these because there are only 8 G\+P\+IB data bus lines.
\begin{DoxyItemize}
\item G\+P\+IB data bus bits are numbered from 1 to 8
\item P\+PR response bits are {\itshape assigned in reverse order} starting from 8, bit 8 for device 0
\begin{DoxyItemize}
\item This is a G\+P\+IB specification -\/ not my idea.
\end{DoxyItemize}
\item The H\+P85 uses these assumptions
\begin{DoxyItemize}
\item P\+PR bits are assigned in reverse order from device numbers.
\end{DoxyItemize}
\end{DoxyItemize}
\end{DoxyItemize}
\end{DoxyItemize}
\item I\+M\+P\+O\+R\+T\+A\+N\+T! On power up the H\+P85 issues a P\+PR query for disk drives
\begin{DoxyItemize}
\item The emulator must be running B\+E\+F\+O\+RE this happens.
\item P\+PR query = both A\+TN and E\+OI being pulled low by the computer.
\item P\+PR response is when each drive pulls a single G\+P\+IB data bus bit L\+OW -\/ while A\+TN and E\+OI are low.
\begin{DoxyItemize}
\item {\itshape O\+N\+LY} those that are detected in this way are then next scanned
\end{DoxyItemize}
\item Next for all detected drives the H\+P85 issues \char`\"{}\+Request Identify\char`\"{} to each in turn.
\begin{DoxyItemize}
\item This is done one drive at a time in order
\item The P\+PR keyword in the \href{sdcard/hpdisk.cfg}{\tt hpdisk.\+cfg} is the P\+PR bit the drive uses
\begin{DoxyItemize}
\item P\+PR of 0 = P\+PR response on G\+P\+IB data bus bit number 8 -\/ as per G\+P\+IB B\+US specifications.
\end{DoxyItemize}
\item The ID keyword in \href{sdcard/hpdisk.cfg}{\tt hpdisk.\+cfg} is the 16 bit reply to \char`\"{}\+Request Identify Reply\char`\"{}
\begin{DoxyItemize}
\item I\+M\+P\+O\+R\+T\+A\+N\+T! A\+M\+I\+GO drives cannot be queried for detailed drive layout information
\begin{DoxyItemize}
\item The H\+P85A can only use its {\itshape hardcoded firmware tables} to map ID to disk layout parameters
\item This implies that the H\+P85A can only use A\+M\+I\+GO disks it has defined in firmware.
\end{DoxyItemize}
\item The H\+P85B can query newer S\+S80 drives for detailed drive layout information instead.
\item The H\+P85A cannot use S\+S80 drives unless it uses copies of the H\+P85B E\+MS and E\+D\+I\+SK R\+O\+MS.
\begin{DoxyItemize}
\item One way this can be done with the P\+R\+M-\/85 expansion board offered by Bill Kotaska
\begin{DoxyItemize}
\item (The P\+R\+M-\/85 is great product giving you access to all of the useful R\+O\+MS) 

 \subsection*{Limitations}
\end{DoxyItemize}
\end{DoxyItemize}
\end{DoxyItemize}
\end{DoxyItemize}
\end{DoxyItemize}
\end{DoxyItemize}

Multiple drive support is impliments but U\+N\+IT support is N\+OT
\begin{DoxyItemize}
\item While most A\+M\+I\+GO and S\+S80 feature have been implemented my primary focus was on the H\+P85A and H\+P85B.
\begin{DoxyItemize}
\item (I do not have access to other computers to test for full compatibility)
\item This means that a few A\+M\+I\+GO and S\+S80 G\+P\+IB commands are not yet implemented!
\begin{DoxyItemize}
\item Some of these are extended reporting modes -\/ many of which are optional.
\end{DoxyItemize}
\item Note\+: The H\+P85A can only use A\+M\+I\+GO drives -\/ unless you have the H\+P85B E\+MS R\+OM installed in your H\+P\+H9A
\begin{DoxyItemize}
\item This can be done with the P\+R\+M-\/85 expansion board offered by Bill Kotaska (a great product!)
\end{DoxyItemize}
\end{DoxyItemize}
\item To attach a drive to our computer, real or otherwise, you must know\+:
\begin{DoxyItemize}
\item The correct G\+P\+IB B\+US address and parallel pool response (P\+PR) bit number your computer expects.
\begin{DoxyItemize}
\item See A\+D\+D\+R\+E\+SS, P\+PR and ID values in \href{sdcard.cfg}{\tt hpdisk.\+cfg}
\end{DoxyItemize}
\item Older computers may only support A\+M\+I\+GO drives.
\begin{DoxyItemize}
\item Such computers will have a hard coded in firmware list of drive its supports.
\begin{DoxyItemize}
\item These computers will issue a G\+P\+IB B\+US \char`\"{}request identify\char`\"{} command and only detect those it knows about.
\item {\itshape If these assumptions do N\+OT match the layout defined in the \href{sdcard/sdcard.cfg}{\tt hpdisk.\+cfg} no drives will be detected.}
\end{DoxyItemize}
\end{DoxyItemize}
\item Newer computers with S\+S80 support can request fully detailed disk layout instead of the \char`\"{}request identify\char`\"{}
\item My emulator supports both reporting methods -\/ but your computer may not use them both!
\begin{DoxyItemize}
\item For supported values consult your computer manuals or corresponding drive manual for your computer.
\begin{DoxyItemize}
\item See gpib/drives\+\_\+parameters.\+txt for a list on some known value (C\+R\+E\+D\+I\+TS; these are from the H\+P\+Dir project)
\end{DoxyItemize}
\item In all cases the \href{sdcard/hpdisk.cfg}{\tt hpdisk.\+cfg} parameters M\+U\+ST match these expectations.
\end{DoxyItemize}
\item The \href{sdcard/hpdisk.cfg}{\tt hpdisk.\+cfg} file tells the emulator how the emulated disk is defined.
\begin{DoxyItemize}
\item G\+P\+IB B\+US address, Parallel Poll Response bit number and A\+M\+I\+GO Request Identify response values.
\item Additional detail for S\+S80 drives that newer computers can use.
\item In A\+LL cases the file informs the code what parameters to emulate and report.
\begin{DoxyItemize}
\item A\+LL of these values M\+U\+ST match your computers expectations for drives it knows about.
\end{DoxyItemize}
\end{DoxyItemize}
\item Debugging
\begin{DoxyItemize}
\item You can enable reporting of all unimplemented G\+P\+IB commands (see {\itshape T\+O\+DO} debug option in \href{sdcard/hpdisk.cfg}{\tt hpdisk.\+cfg} )
\begin{DoxyItemize}
\item Useful if you are trying this on a non H\+P85 device
\item See the \href{sdcard/hpdisk.cfg}{\tt hpdisk.\+cfg} for documentation on the full list of debugging options
\end{DoxyItemize}
\item The emulator can passively log all transactions between real hardware on the G\+P\+IB bus
\begin{DoxyItemize}
\item Use the \char`\"{}gpib trace $\ast$logfile$\ast$\char`\"{} command -\/ pressing any key exits -\/ no emulation is done in this mode.
\item You can use this to help understand what is sent to and from your real disks.
\item I use this feature to help prioritize which commands I first implemented. 


\end{DoxyItemize}
\end{DoxyItemize}
\end{DoxyItemize}
\end{DoxyItemize}

\subsection*{Credits}

{\bfseries I really owe the very existence of this project to the original work done by Anders Gustafsson and his great \char`\"{}\+H\+P Disk Emulator\char`\"{} }
\begin{DoxyItemize}
\item You can visit his project at this site\+:
\begin{DoxyItemize}
\item \href{http://www.dalton.ax/hpdisk}{\tt http\+://www.\+dalton.\+ax/hpdisk}
\item \href{http://www.elektor-labs.com/project/hpdisk-an-sd-based-disk-emulator-for-gpib-instruments-and-computers.13693.html}{\tt http\+://www.\+elektor-\/labs.\+com/project/hpdisk-\/an-\/sd-\/based-\/disk-\/emulator-\/for-\/gpib-\/instruments-\/and-\/computers.\+13693.\+html}
\end{DoxyItemize}
\end{DoxyItemize}

{\bfseries  The H\+P\+Dir project was vital as a documentation source for this project}
\begin{DoxyItemize}
\item \href{http://www.hp9845.net/9845/projects/hpdir}{\tt http\+://www.\+hp9845.\+net/9845/projects/hpdir}
\end{DoxyItemize}

{\bfseries Anders Gustafsson was extremely helpful in providing me his current code and details of his project -\/ which I am very thankful for.}

As mainly a personal exercise in fully understanding the code I ended up rewriting much of the hpdisk project. I did this one part at a time as I learned the protocols and specifications -\/ N\+OT because of any problems with his original work.

Although mostly rewritten I have maintained the basic concept of using state machines for G\+P\+IB read and write functions as well as for S\+S80 execute state tracking.

\href{lif/teledisk}{\tt lif/teledisk}
\begin{DoxyItemize}
\item \href{lif/teledisk}{\tt lif/teledisk}
\begin{DoxyItemize}
\item My T\+E\+L\+E\+D\+I\+SK L\+IF extracter
\item Important Contributions (My converted would not have been possible without these)
\begin{DoxyItemize}
\item Dave Dunfield, L\+Z\+SS Code and Tele\+Disk documentation
\begin{DoxyItemize}
\item Copyright 2007-\/2008 Dave Dunfield All rights reserved.
\item \href{lif/teledisk/td0_lzss.h}{\tt td0\+\_\+lzss.\+h}
\item \href{lif/teledisk/td0_lzss.c}{\tt td0\+\_\+lzss.\+c}
\begin{DoxyItemize}
\item L\+Z\+SS decoder
\end{DoxyItemize}
\item \href{lif/teledisk/td0notes.txt}{\tt td0notes.\+txt}
\begin{DoxyItemize}
\item Teledisk Documentation
\end{DoxyItemize}
\end{DoxyItemize}
\item Jean-\/\+Franois D\+EL N\+E\+RO, Tele\+Disk Documentation
\begin{DoxyItemize}
\item Copyright (C) 2006-\/2014 Jean-\/\+Franois D\+EL N\+E\+RO
\begin{DoxyItemize}
\item \href{lif/teledisk/wteledsk.htm}{\tt wteledsk.\+htm}
\begin{DoxyItemize}
\item Tele\+Disk documenation
\end{DoxyItemize}
\item See his github project
\begin{DoxyItemize}
\item \href{https://github.com/jfdelnero/libhxcfe}{\tt https\+://github.\+com/jfdelnero/libhxcfe} 

 \section*{Abbreviations}
\end{DoxyItemize}

Within this project I have attempted to provide detailed references to manuals, listed below. I have included short quotes and section and page\# reference to these works.
\end{DoxyItemize}
\end{DoxyItemize}
\end{DoxyItemize}
\end{DoxyItemize}
\item {\bfseries S\+S80}
\item {\bfseries C\+S80}
\item {\bfseries A or Amigo}
\item {\bfseries H\+P-\/\+IP}
\item {\bfseries H\+P-\/\+IP Tutorial}
\end{DoxyItemize}



 \subsection*{Documentation References and related sources of information}


\begin{DoxyItemize}
\item Web Resources
\begin{DoxyItemize}
\item \href{http://www.hp9845.net}{\tt http\+://www.\+hp9845.\+net}
\item \href{http://www.hpmuseum.net}{\tt http\+://www.\+hpmuseum.\+net}
\item \href{http://www.hpmusuem.org}{\tt http\+://www.\+hpmusuem.\+org}
\item \href{http://bitsavers.trailing-edge.com}{\tt http\+://bitsavers.\+trailing-\/edge.\+com}
\item \href{http://en.wikipedia.org/wiki/IEEE-488}{\tt http\+://en.\+wikipedia.\+org/wiki/\+I\+E\+E\+E-\/488}
\item See \mbox{[}Documents folder\mbox{]}(documents)
\end{DoxyItemize}
\end{DoxyItemize}



 \subsection*{Enhanced version of Tony Duell\textquotesingle{}s lif\+\_\+utils by Joachim}


\begin{DoxyItemize}
\item \href{https://github.com/bug400/lifutils}{\tt https\+://github.\+com/bug400/lifutils}
\item Create/\+Modify L\+IF images
\end{DoxyItemize}



 \subsection*{C\+S80 References\+: (\char`\"{}\+C\+S80\char`\"{} is the short form used in the project)}


\begin{DoxyItemize}
\item \char`\"{}\+C\+S/80 Instruction Set Programming Manual\char`\"{}
\item Printed\+: A\+PR 1983
\item HP Part\# 5955-\/3442
\item See \mbox{[}Documents folder\mbox{]}(documents)
\end{DoxyItemize}



 \subsection*{Amigo References\+: (\char`\"{}\+A\char`\"{} or \char`\"{}\+Amigo\char`\"{} is the short form used in the project)}


\begin{DoxyItemize}
\item \char`\"{}\+Appendix A of 9895\+A Flexible Disc Memory Service Manual\char`\"{}
\item HP Part\# 09895-\/90030
\item See \mbox{[}Documents folder\mbox{]}(documents)
\end{DoxyItemize}



 \subsection*{H\+P-\/\+IB}


\begin{DoxyItemize}
\item (\char`\"{}\+H\+P-\/\+I\+B\char`\"{} is the short form used in the project)
\item \char`\"{}\+Condensed Description of the Hewlett Packard Interface Bus\char`\"{}
\item Printed March 1975
\item HP Part\# 59401-\/90030
\item See \mbox{[}Documents folder\mbox{]}(documents)
\end{DoxyItemize}



 \subsection*{Tutorial Description of the Hewlett Packard Interface Bus}


\begin{DoxyItemize}
\item (\char`\"{}\+H\+P-\/\+I\+B Tutorial\char`\"{} is the short form used in the project)
\item \href{http://www.hpmemory.org/an/pdf/hp-ib_tutorial_1980.pdf}{\tt http\+://www.\+hpmemory.\+org/an/pdf/hp-\/ib\+\_\+tutorial\+\_\+1980.\+pdf}
\item Printed January 1983
\item \href{http://www.ko4bb.com/Manuals/HP_Agilent/HPIB_tutorial_HP.pdf}{\tt http\+://www.\+ko4bb.\+com/\+Manuals/\+H\+P\+\_\+\+Agilent/\+H\+P\+I\+B\+\_\+tutorial\+\_\+\+H\+P.\+pdf}
\item Printed 1987
\item See \mbox{[}Documents folder\mbox{]}(documents)
\end{DoxyItemize}



 \subsection*{G\+P\+IB / I\+E\+EE 488 Tutorial by Ian Poole}


\begin{DoxyItemize}
\item \href{http://www.radio-electronics.com/info/t_and_m/gpib/ieee488-basics-tutorial.php}{\tt http\+://www.\+radio-\/electronics.\+com/info/t\+\_\+and\+\_\+m/gpib/ieee488-\/basics-\/tutorial.\+php}
\end{DoxyItemize}

See \mbox{[}Documents folder\mbox{]}(documents)



 \subsection*{HP 9133/9134 D/\+H/L References}


\begin{DoxyItemize}
\item \char`\"{}\+H\+P 9133/9134 D/\+H/\+L Service Manual\char`\"{}
\item HP Part\# 5957-\/6560
\item Printed\+: A\+P\+R\+IL 1985, Edition 2
\item See \mbox{[}Documents folder\mbox{]}(documents) 


\end{DoxyItemize}

\subsection*{L\+IF File system Format}


\begin{DoxyItemize}
\item \href{http://www.hp9845.net/9845/projects/hpdir/#lif_filesystem}{\tt http\+://www.\+hp9845.\+net/9845/projects/hpdir/\#lif\+\_\+filesystem}
\item See \mbox{[}Documents folder\mbox{]}(documents) 


\end{DoxyItemize}

\subsection*{Useful Utilities}


\begin{DoxyItemize}
\item HP Drive (HP Drive Emulators for Windows Platform)
\begin{DoxyItemize}
\item \href{http://www.hp9845.net/9845/projects/hpdrive/}{\tt http\+://www.\+hp9845.\+net/9845/projects/hpdrive/}
\end{DoxyItemize}
\item HP Dir (HP Drive -\/ Disk Image Manipulation)
\begin{DoxyItemize}
\item \href{http://www.hp9845.net/9845/projects/hpdir/}{\tt http\+://www.\+hp9845.\+net/9845/projects/hpdir/} 


\end{DoxyItemize}
\end{DoxyItemize}

\subsection*{G\+P\+IB Connector pinout by Anders Gustafsson in his hpdisk project}


\begin{DoxyItemize}
\item \href{http://www.dalton.ax/hpdisk/}{\tt http\+://www.\+dalton.\+ax/hpdisk/}
\end{DoxyItemize}


\begin{DoxyPre}
    Pin Name   Signal Description       Pin Name   Signal Description 
    1   DIO1   Data Input/Output Bit 1  13  DIO5   Data Input/Output Bit 5 
    2   DIO2   Data Input/Output Bit 2  14  DIO6   Data Input/Output Bit 6 
    3   DIO3   Data Input/Output Bit 3  15  DIO7   Data Input/Output Bit 7 
    4   DIO4   Data Input/Output Bit 4  16  DIO8   Data Input/Output Bit 8 
    5   EIO    End-Or-Identify          17  REN    Remote Enable 
    6   DAV    Data Valid               18  Shield Ground (DAV) 
    7   NRFD   Not Ready For Data       19  Shield Ground (NRFD) 
    8   NDAC   Not Data Accepted        20  Shield Ground (NDAC) 
    9   IFC    Interface Clear          21  Shield Ground (IFC) 
    10  SRQ    Service Request          22  Shield Ground (SRQ) 
    11  ATN    Attention                23  Shield Ground (ATN) 
    12  Shield Chassis Ground           24  Single GND Single Ground
\end{DoxyPre}


\subsection*{Testing}


\begin{DoxyItemize}
\item Testing was done with an H\+P85A (with extended E\+MS R\+OM)
\begin{DoxyItemize}
\item Using the Hewlett-\/\+Packard Series 80 -\/ P\+R\+M-\/85 by Bill Kotaska
\item This makes my H\+P85A look like and H\+P85B
\begin{DoxyItemize}
\item I can also use the normal mass storage R\+OM if I limit to A\+M\+I\+GO drives.
\item \href{http://vintagecomputers.sdfeu.org/hp85/prm85.htm}{\tt http\+://vintagecomputers.\+sdfeu.\+org/hp85/prm85.\+htm}
\begin{DoxyItemize}
\item old site \href{http://vintagecomputers.site90.net/hp85/prm85.htm}{\tt http\+://vintagecomputers.\+site90.\+net/hp85/prm85.\+htm}
\end{DoxyItemize}
\end{DoxyItemize}
\end{DoxyItemize}
\item Note\+: the E\+MS R\+OM has extended I\+N\+I\+T\+I\+A\+L\+I\+ZE attributes 
\begin{DoxyPre}
  \# Initializing: (already done on these images so you do not have to)
  INITIALIZE "SS80-1",":D700",128,1
  INITIALIZE "AMIGO1",":D710",14,1
  INITIALIZE "SS80-2",":D720",128,1
  INITIALIZE "AMIGO2",":D730",14,1\end{DoxyPre}

\end{DoxyItemize}


\begin{DoxyPre}  \# Listing files:
  \# first AMIGO
  CAT ":D700"
  \# second AMIGO
  CAT ":D710"
  \# first SS80
  CAT ":D720"
  \# second SS*0
  CAT ":D730"\end{DoxyPre}



\begin{DoxyPre}  \# Loading file from second SS80:
  LOAD "HELLO:D720"
  \# Copying file between devices: fist AMIGO to second AMIGO
  COPY "HELLO:D710" TO "HELLO:D730"
  \# Copying ALL files between devices: FIRST SS80 to Second SS80
  COPY ":D700" TO ":D720"\end{DoxyPre}



\begin{DoxyPre}  \# Working with plain text files and the HP85
  \# How to import and export plain text, ascii, files 
  \# My lif tools can read or write HP85 files that GETSAVE can then read and write
  \# GETSAVE is included in all of my supplied disk images
  \#   The lif tools are included in the firmwarei and you can use then from a terminal program
  \#   1) My lif tools are built in the firmware
  \#   2) GETSAVE is included in ALL emulator disk images under the [sdcard](sdcard) folder
  \#      GETSAVE and read and write TYPE E010 files
  \# Usesge:
  LOADBIN "GETSAVE"
  \# This program stays in memory until the HP85 is reset
  \# How do we add an Ascii text file to a disk emulator image?
    \# Connect the emulator to a serial terminal - see the [Makefile](Makefile) for the default baud rate
    \# The baud rate is at the begining of the Makefile
    \# Connect to the disk emulator with a serial terminal
  \# Adding text file to the sdcard
    \# Lets say we created a pain text file containing a basic program call test.txt
    \# With the HP85 and emulator powerd off copy it to the sdcard
    \# Now put the sdcard back in the emulator and power it on - then turn on your HP85 last
  \# Adding the text file to the emulator image - lets use emulator image amigo1.lif
  lif add amigo1.lif MYTEST basic.txt
  \# We just added the file to the image file called amigo1.lif and named it MYTEST
  \# Lets assume amigo1.lif is defined as device :D700 in the hpdisk.cfg file
  GET "MYTEST:D700"
  \# Save it 
  PUT "MYTEST2:D700"
  \# Lets store it as a program se we do not have to use the concersion tools
  STORE "MYTESTB:D700"
  \# In the future we can LOAD it without the conversion or GETSAVE program
  LOAD "MYTESTB:D700"\end{DoxyPre}



\begin{DoxyPre}  \# Only available if you have advanced EMS and electronic disk roms
  \# How to Delete a file
  PURGE "HELLO:D730"
  \# Load a BASIC format program
  LOAD "HELLO:D700"
  \# Save a BASIC format program
  SAVE "HELLO:D710"
  \# Clear memory
  SCRATCH
  \# List a BASIC program
  LIST
\end{DoxyPre}
 



\subsection*{A\+VR Terminal Commands}


\begin{DoxyItemize}
\item Pressing any key will break out of the gpib task loop until a command is entered
\begin{DoxyItemize}
\item help
\begin{DoxyItemize}
\item Will list all available commands and options
\item Each main option has help. Example lif help
\end{DoxyItemize}
\item For detail using L\+IF commands to add/extract L\+IF files from SD card see the top of \href{lif/lifutil.c}{\tt lif/lifutil.\+c}
\end{DoxyItemize}
\end{DoxyItemize}





\subsection*{Main project files for hp85disk Project}


\begin{DoxyItemize}
\item Project Main Home Directory
\begin{DoxyItemize}
\item \href{main.c}{\tt main.\+c}
\item \href{main.h}{\tt main.\+h}
\begin{DoxyItemize}
\item Main start-\/up code
\end{DoxyItemize}
\item \mbox{[}Makefile\mbox{]}(Makefile)
\begin{DoxyItemize}
\item Main Project Makefile
\end{DoxyItemize}
\end{DoxyItemize}
\item Terminal scripts
\begin{DoxyItemize}
\item \mbox{[}miniterm\mbox{]}(miniterm)
\begin{DoxyItemize}
\item wrapper for miniterm.\+py part of the python package pyserial
\end{DoxyItemize}
\item \mbox{[}term\mbox{]}(term)
\begin{DoxyItemize}
\item Wrapper for minicom terminal emulator
\end{DoxyItemize}
\end{DoxyItemize}
\item Doxygen files
\begin{DoxyItemize}
\item \mbox{[}Doxyfile\mbox{]}(Doxyfile)
\begin{DoxyItemize}
\item Doxygen Configuration files for project
\end{DoxyItemize}
\item \mbox{[}doxyinc\mbox{]}(doxyinc)
\begin{DoxyItemize}
\item Determins which files are included in the project Doxygen documents
\end{DoxyItemize}
\item \href{DoxygenLayout.xml}{\tt Doxygen\+Layout.\+xml}
\begin{DoxyItemize}
\item Doxygen Layout file
\end{DoxyItemize}
\end{DoxyItemize}
\item Project Readme
\begin{DoxyItemize}
\item \hyperlink{md_README}{R\+E\+A\+D\+ME.md}
\begin{DoxyItemize}
\item Project R\+E\+A\+D\+ME
\end{DoxyItemize}
\end{DoxyItemize}
\item Project Copyright
\begin{DoxyItemize}
\item \hyperlink{COPYRIGHT_8md}{C\+O\+P\+Y\+R\+I\+G\+HT.md}
\begin{DoxyItemize}
\item Project Copy\+Rights
\end{DoxyItemize}
\end{DoxyItemize}
\end{DoxyItemize}

\subsection*{Board design file for version 1 and 2 hardware information}


\begin{DoxyItemize}
\item \mbox{[}board\mbox{]}(board)
\begin{DoxyItemize}
\item \href{board/V1}{\tt V1}
\begin{DoxyItemize}
\item V1 Board documentation and Release files
\item \href{board/V1/HP85Disk.pdf}{\tt board design and pinouts of this project and a schematic P\+DF}
\item \href{board/V1//HP85Disk.doc}{\tt board design and pinouts of this project and a schematic D\+OC}
\item \href{board/V1/HP85Disk.doc}{\tt board R\+E\+A\+D\+M\+E.\+md}
\end{DoxyItemize}
\item \href{V2/releses}{\tt V2/releases}
\begin{DoxyItemize}
\item Jay Hamlin version 2 circuit board design using G\+P\+IB buffers
\end{DoxyItemize}
\end{DoxyItemize}
\end{DoxyItemize}

\subsection*{Documents}


\begin{DoxyItemize}
\item \mbox{[}Documents\mbox{]}(Documents)
\item G\+P\+IB B\+US, HP device, L\+IF and chips documentation for this project
\begin{DoxyItemize}
\item Documents/\+R\+E\+A\+D\+ME.md
\end{DoxyItemize}
\end{DoxyItemize}

\subsection*{Fat\+Fs}


\begin{DoxyItemize}
\item \mbox{[}fatfs\mbox{]}(fatfs)
\begin{DoxyItemize}
\item R0.\+12b Fat\+FS code from (C) ChaN, 2016 -\/ With very minimal changes
\item \href{fatfs/00history.txt}{\tt 00history.\+txt}
\item \href{fatfs/00readme.txt}{\tt 00readme.\+txt}
\item \href{fatfs/ff.c}{\tt ff.\+c}
\item \href{fatfs/ffconf.h}{\tt ffconf.\+h}
\item \href{fatfs/ff.h}{\tt ff.\+h}
\item \href{fatfs/integer.h}{\tt integer.\+h}
\end{DoxyItemize}
\item \href{fatfs.hal/fatfs.hal}{\tt fatfs.\+hal}
\begin{DoxyItemize}
\item R0.\+12b Fat\+FS code from (C) ChaN, 2016 with changes
\begin{DoxyItemize}
\item Hardware abstraction layer based on example A\+VR project
\end{DoxyItemize}
\item \href{fatfs.hal/diskio.c}{\tt diskio.\+c}
\begin{DoxyItemize}
\item Low level disk I/O module glue functions (fatfs.\+hal/C)ChaN, 2016
\end{DoxyItemize}
\item \href{fatfs.hal/diskio.h}{\tt diskio.\+h}
\begin{DoxyItemize}
\item Low level disk I/O module glue functions (fatfs.\+hal/C)ChaN, 2016
\end{DoxyItemize}
\item \href{fatfs.hal/mmc.c}{\tt mmc.\+c}
\begin{DoxyItemize}
\item Low level M\+MC I/O by (fatfs.\+hal/C)ChaN, 2016 with modifications
\end{DoxyItemize}
\item \href{fatfs.hal/mmc.h}{\tt mmc.\+h}
\begin{DoxyItemize}
\item Low level M\+MC I/O by (fatfs.\+hal/C)ChaN, 2016 with modifications
\end{DoxyItemize}
\item \href{fatfs.hal/mmc_hal.c}{\tt mmc\+\_\+hal.\+c}
\begin{DoxyItemize}
\item My Hardware abstraction layer code
\end{DoxyItemize}
\item \href{fatfs.hal/mmc_hal.h}{\tt mmc\+\_\+hal.\+h}
\begin{DoxyItemize}
\item My Hardware abstraction layer code
\end{DoxyItemize}
\end{DoxyItemize}
\item \href{fatfs.sup/fatfs.sup}{\tt fatfs.\+sup}
\begin{DoxyItemize}
\item Support utility and P\+O\+S\+IX wrapper functions
\item \href{fatfs.sup/fatfs.h}{\tt fatfs.\+h}
\begin{DoxyItemize}
\item Fat\+FS header files
\end{DoxyItemize}
\item \href{fatfs.sup/fatfs_sup.c}{\tt fatfs\+\_\+sup.\+c}
\item \href{fatfs.sup/fatfs_sup.h}{\tt fatfs\+\_\+sup.\+h}
\begin{DoxyItemize}
\item Fat\+FS file listing and display functions
\end{DoxyItemize}
\item \href{fatfs.sup/fatfs_tests.c}{\tt fatfs\+\_\+tests.\+c}
\item \href{fatfs.sup/fatfs_tests.h}{\tt fatfs\+\_\+tests.\+h}
\begin{DoxyItemize}
\item Fat\+FS user test functions
\end{DoxyItemize}
\end{DoxyItemize}
\end{DoxyItemize}

\subsection*{G\+P\+IB related}


\begin{DoxyItemize}
\item \href{gpib/gpib}{\tt gpib}
\begin{DoxyItemize}
\item My G\+P\+IB code for A\+M\+I\+GO S\+S80 and P\+P\+R\+I\+N\+T\+ER support
\item \href{gpib/amigo.c}{\tt amigo.\+c}
\begin{DoxyItemize}
\item A\+M\+I\+GO parser
\end{DoxyItemize}
\item \href{gpib/amigo.h}{\tt amigo.\+h}
\begin{DoxyItemize}
\item A\+M\+I\+GO parser
\end{DoxyItemize}
\item \href{gpib/defines.h}{\tt defines.\+h}
\begin{DoxyItemize}
\item Main G\+P\+IB header and configuration options
\end{DoxyItemize}
\item \href{debug.txt}{\tt debug.\+txt}
\begin{DoxyItemize}
\item List of debug flags
\end{DoxyItemize}
\item \href{gpib/drives.c}{\tt drives.\+c}
\begin{DoxyItemize}
\item Supported Drive Parameters
\end{DoxyItemize}
\item \href{gpib/drive_references.txt}{\tt drive\+\_\+references.\+txt}
\begin{DoxyItemize}
\item General Drive Parameters Documentation for all known drive types
\end{DoxyItemize}
\item \href{gpib/format.c}{\tt format.\+c}
\begin{DoxyItemize}
\item L\+IF format and file utilities
\end{DoxyItemize}
\item \href{gpib/gpib.c}{\tt gpib.\+c}
\begin{DoxyItemize}
\item All low level G\+P\+IB bus code
\end{DoxyItemize}
\item \href{gpib/gpib.h}{\tt gpib.\+h}
\begin{DoxyItemize}
\item G\+P\+IB I/O code
\end{DoxyItemize}
\item \href{gpib/gpib_hal.c}{\tt gpib\+\_\+hal.\+c}
\begin{DoxyItemize}
\item G\+P\+IB hardware abstraction code
\end{DoxyItemize}
\item \href{gpib/gpib_hal.h}{\tt gpib\+\_\+hal.\+h}
\begin{DoxyItemize}
\item G\+P\+IB hardware abstraction code
\end{DoxyItemize}
\item \href{gpib/gpib_task.c}{\tt gpib\+\_\+task.\+c}
\begin{DoxyItemize}
\item G\+P\+IB command handler , initialization and tracing code
\end{DoxyItemize}
\item \href{gpib/gpib_task.h}{\tt gpib\+\_\+task.\+h}
\begin{DoxyItemize}
\item G\+P\+IB command handler , initialization and tracing code
\end{DoxyItemize}
\item \href{gpib/gpib_tests.c}{\tt gpib\+\_\+tests.\+c}
\begin{DoxyItemize}
\item G\+P\+IB user tests
\end{DoxyItemize}
\item \href{gpib/gpib_tests.h}{\tt gpib\+\_\+tests.\+h}
\begin{DoxyItemize}
\item G\+P\+IB user tests
\end{DoxyItemize}
\item \href{gpib/printer.c}{\tt printer.\+c}
\begin{DoxyItemize}
\item G\+P\+IB printer capture code
\end{DoxyItemize}
\item \href{gpib/printer.h}{\tt printer.\+h}
\begin{DoxyItemize}
\item G\+P\+IB printer capture code
\end{DoxyItemize}
\item \href{gpib/references.txt}{\tt references.\+txt}
\begin{DoxyItemize}
\item Main S80 S\+S80 A\+M\+I\+GO and G\+P\+IB references part numbers and web links
\end{DoxyItemize}
\item \href{gpib/ss80.c}{\tt ss80.\+c}
\begin{DoxyItemize}
\item S\+S80 parser
\end{DoxyItemize}
\item \href{gpib/ss80.h}{\tt ss80.\+h}
\begin{DoxyItemize}
\item S\+S80 parser
\end{DoxyItemize}
\item \href{gpib/notes.txt}{\tt notes.\+txt}
\begin{DoxyItemize}
\item My notes on G\+P\+IB bus states as it relates to the project
\end{DoxyItemize}
\end{DoxyItemize}
\end{DoxyItemize}

\subsection*{Hardware C\+PU specific}


\begin{DoxyItemize}
\item \mbox{[}hardware\mbox{]}(hardware)
\begin{DoxyItemize}
\item C\+PU hardware specific code
\item \href{hardware/baudrate.c}{\tt baudrate.\+c}
\begin{DoxyItemize}
\item Baud rate calculation tool. Given C\+PU clock and desired baud rate, will list the actual baud rate and registers
\end{DoxyItemize}
\item \href{hardware/bits.h}{\tt bits.\+h}
\begin{DoxyItemize}
\item B\+IT set and clear functions
\end{DoxyItemize}
\item \href{hardware/cpu.h}{\tt cpu.\+h}
\begin{DoxyItemize}
\item C\+PU specific include files
\end{DoxyItemize}
\item \href{hardware/delay.c}{\tt delay.\+c}
\item \href{hardware/delay.h}{\tt delay.\+h}
\begin{DoxyItemize}
\item Delay code
\end{DoxyItemize}
\item \href{hardware/hal.c}{\tt hal.\+c}
\item \href{hardware/hal.h}{\tt hal.\+h}
\begin{DoxyItemize}
\item G\+P\+IO functions, spi hardware abstraction layer and chip select logic
\end{DoxyItemize}
\item \href{hardware/iom1284p.h}{\tt iom1284p.\+h}
\begin{DoxyItemize}
\item G\+P\+IO map for A\+T\+E\+M\+E\+GA 1284p
\end{DoxyItemize}
\item \href{hardware/mkdef.c}{\tt mkdef.\+c}
\begin{DoxyItemize}
\item Not used
\end{DoxyItemize}
\item \href{hardware/pins.txt}{\tt pins.\+txt}
\begin{DoxyItemize}
\item A\+VR function to G\+P\+IO pin map
\end{DoxyItemize}
\item \href{hardware/ram.c}{\tt ram.\+c}
\item \href{hardware/ram.h}{\tt ram.\+h}
\begin{DoxyItemize}
\item Memory functions
\end{DoxyItemize}
\item \href{hardware/rs232.c}{\tt rs232.\+c}
\item \href{hardware/rs232.h}{\tt rs232.\+h}
\begin{DoxyItemize}
\item R\+S232 IO
\end{DoxyItemize}
\item \href{hardware/rtc.c}{\tt rtc.\+c}
\item \href{hardware/rtc.h}{\tt rtc.\+h}
\begin{DoxyItemize}
\item D\+S1307 I2C R\+TC code
\end{DoxyItemize}
\item \href{hardware/spi.c}{\tt spi.\+c}
\item \href{hardware/spi.h}{\tt spi.\+h}
\begin{DoxyItemize}
\item S\+PI B\+US code
\end{DoxyItemize}
\item \href{hardware/TWI_AVR8.c}{\tt T\+W\+I\+\_\+\+A\+V\+R8.\+c}
\item \href{hardware/TWI_AVR8.h}{\tt T\+W\+I\+\_\+\+A\+V\+R8.\+h}
\begin{DoxyItemize}
\item I2C code L\+U\+FA Library Copyright (hardware/C) Dean Camera, 2011.
\end{DoxyItemize}
\item \href{hardware/user_config.h}{\tt user\+\_\+config.\+h}
\begin{DoxyItemize}
\item Main include file M\+MC S\+L\+OW and F\+A\+TS frequency and C\+PU frequency settings
\end{DoxyItemize}
\end{DoxyItemize}
\end{DoxyItemize}

\subsection*{Common libraries}


\begin{DoxyItemize}
\item \mbox{[}lib\mbox{]}(lib)
\begin{DoxyItemize}
\item Library functions
\item \href{lib/bcpp.cfg}{\tt bcpp.\+cfg}
\begin{DoxyItemize}
\item B\+C\+PP C code formatter config
\end{DoxyItemize}
\item \href{lib/matrix.c}{\tt matrix.\+c}
\item \href{lib/matrix.h}{\tt matrix.\+h}
\begin{DoxyItemize}
\item Matrix code -\/ not used
\end{DoxyItemize}
\item \href{lib/matrix.txt}{\tt matrix.\+txt}
\begin{DoxyItemize}
\item A few notes about matrix operations
\end{DoxyItemize}
\item \href{lib/queue.c}{\tt queue.\+c}
\item \href{lib/queue.h}{\tt queue.\+h}
\begin{DoxyItemize}
\item Queue functions
\end{DoxyItemize}
\item \href{lib/sort.c}{\tt sort.\+c}
\item \href{lib/sort.h}{\tt sort.\+h}
\begin{DoxyItemize}
\item Sort functions -\/ not used
\end{DoxyItemize}
\item \href{lib/stringsup.c}{\tt stringsup.\+c}
\item \href{lib/stringsup.h}{\tt stringsup.\+h}
\begin{DoxyItemize}
\item Various string processing functions
\end{DoxyItemize}
\item \href{lib/time.c}{\tt time.\+c}
\item \href{lib/time.h}{\tt time.\+h}
\begin{DoxyItemize}
\item P\+O\+S\+IX time functions
\end{DoxyItemize}
\item \href{lib/timer.c}{\tt timer.\+c}
\item \href{lib/timer.h}{\tt timer.\+h}
\begin{DoxyItemize}
\item Timer task functions
\end{DoxyItemize}
\item \href{lib/timer_hal.c}{\tt timer\+\_\+hal.\+c}
\begin{DoxyItemize}
\item Timer task hardware abstraction layer
\end{DoxyItemize}
\item \href{lib/timetests.c}{\tt timetests.\+c}
\begin{DoxyItemize}
\item Time and timer test code
\end{DoxyItemize}
\end{DoxyItemize}
\end{DoxyItemize}

\subsection*{L\+IF files}


\begin{DoxyItemize}
\item \mbox{[}lif\mbox{]}(lif)
\begin{DoxyItemize}
\item L\+IF disk image utilities
\item \href{lif/lifutils.c}{\tt lif/lifutils.\+c}
\item \href{lif/lifutils.c}{\tt lif/lifutils.\+c}
\begin{DoxyItemize}
\item Functions that allow the emulator to import and export files from L\+IF images
\end{DoxyItemize}
\item \href{lif/Makefile}{\tt Makefile}
\begin{DoxyItemize}
\item Permits creating a standalone Linux version of the L\+IF emulator tools
\end{DoxyItemize}
\item Code by Mike Gore
\begin{DoxyItemize}
\item \href{lif/Makefile}{\tt Makefile}
\begin{DoxyItemize}
\item Make stand alone Linux versions of L\+IF utility and optionaly Tele\+Disk to L\+IF converter
\end{DoxyItemize}
\item \href{lif/lifsup.c}{\tt lifsup.\+c}
\item \href{lif/lifsup.h}{\tt lifsup.\+h}
\begin{DoxyItemize}
\item Stand alone libraries for L\+IF and T\+E\+L\+E\+D\+I\+SK utility
\begin{DoxyItemize}
\item These functions are also part of various hp85disk libraries
\end{DoxyItemize}
\end{DoxyItemize}
\item \href{lif/lifutils.c}{\tt lifutils.\+c}
\item \href{lif/lifutils.h}{\tt lifutils.\+h}
\begin{DoxyItemize}
\item L\+IF image functions, directory listing and file adding.\+extracting,renaming,deleting
\end{DoxyItemize}
\item \href{lif/td02lif.c}{\tt td02lif.\+c}
\item \href{lif/td02lif.h}{\tt td02lif.\+h}
\begin{DoxyItemize}
\item My Tele\+Disk to L\+IF translator
\end{DoxyItemize}
\item \href{lif/lif-notes.txt}{\tt lif-\/notes.\+txt}
\begin{DoxyItemize}
\item My notes on decoding E010 format L\+IF images for H\+P-\/85
\end{DoxyItemize}
\item \href{lif/README.txt}{\tt R\+E\+A\+D\+M\+E.\+txt}
\begin{DoxyItemize}
\item Notes on each file under L\+IF and lif/teledisk
\end{DoxyItemize}
\item \href{lif/85-SS80.TD0}{\tt 85-\/\+S\+S80.\+T\+D0} from hpmuseum.\+org
\begin{DoxyItemize}
\item Damaged S\+S80 Excersizer from HP Musium
\end{DoxyItemize}
\item \href{lif/85-SS80.LIF}{\tt 85-\/\+S\+S80.\+L\+IF}
\begin{DoxyItemize}
\item The current converter automaticcal did these repairs
\begin{DoxyItemize}
\item cyl 11, side 0 sector 116 mapped to 8
\item cyl 13, side 0 sector 116 mapped to 11
\item cyl 15, side 0 sector 10 missing -\/ zero filled
\end{DoxyItemize}
\end{DoxyItemize}
\end{DoxyItemize}
\end{DoxyItemize}
\end{DoxyItemize}

\subsection*{L\+IF teledisk files}


\begin{DoxyItemize}
\item \href{lif/teledisk}{\tt lif/teledisk}
\begin{DoxyItemize}
\item My T\+E\+L\+E\+D\+I\+SK L\+IF extracter
\begin{DoxyItemize}
\item Note\+: The Tele\+Disk image M\+U\+ST contain a L\+IF image -\/ we do N\+OT translate it
\end{DoxyItemize}
\item \href{lif/teledisk/README.txt}{\tt R\+E\+A\+D\+M\+E.\+txt}
\begin{DoxyItemize}
\item Credits
\end{DoxyItemize}
\item Important Contributions (My converter would not have been possible without these)
\begin{DoxyItemize}
\item Dave Dunfield, L\+Z\+SS Code and Tele\+Disk documentation
\begin{DoxyItemize}
\item Copyright 2007-\/2008 Dave Dunfield All rights reserved.
\item \href{lif/teledisk/td0_lzss.h}{\tt td0\+\_\+lzss.\+h}
\item \href{lif/teledisk/td0_lzss.c}{\tt td0\+\_\+lzss.\+c}
\begin{DoxyItemize}
\item L\+Z\+SS decoder
\end{DoxyItemize}
\item \href{lif/teledisk/td0notes.txt}{\tt td0notes.\+txt}
\begin{DoxyItemize}
\item Teledisk Documentation
\end{DoxyItemize}
\end{DoxyItemize}
\item Jean-\/\+Franois D\+EL N\+E\+RO, Tele\+Disk Documentation
\begin{DoxyItemize}
\item Copyright (lif/teledisk/C) 2006-\/2014 Jean-\/\+Franois D\+EL N\+E\+RO
\begin{DoxyItemize}
\item \href{lif/teledisk/wteledsk.htm}{\tt wteledsk.\+htm}
\begin{DoxyItemize}
\item Tele\+Disk documenation
\end{DoxyItemize}
\item See his github project
\begin{DoxyItemize}
\item \href{https://github.com/jfdelnero/libhxcfe}{\tt https\+://github.\+com/jfdelnero/libhxcfe}
\end{DoxyItemize}
\end{DoxyItemize}
\end{DoxyItemize}
\end{DoxyItemize}
\end{DoxyItemize}
\end{DoxyItemize}

\subsection*{Posix wrapper -\/ provides linux file IO functions for Fatfs}


\begin{DoxyItemize}
\item \mbox{[}posix\mbox{]}(posix)
\begin{DoxyItemize}
\item P\+O\+S\+IX wrappers provide many U\+N\+IX P\+O\+S\+IX compatible functions by translating fatfs functions
\item \href{buffer.c}{\tt buffer.\+c}
\item \href{buffer.h}{\tt buffer.\+h}
\begin{DoxyItemize}
\item Currently unused in this project
\end{DoxyItemize}
\item \href{posix/posix.c}{\tt posix.\+c}
\item \href{posix/posix.h}{\tt posix.\+h}
\begin{DoxyItemize}
\item P\+O\+S\+IX wrappers for fatfs -\/ Unix file IO function call wrappers
\end{DoxyItemize}
\item \href{posix/posix_tests.c}{\tt posix\+\_\+tests.\+c}
\item \href{posix/posix_tests.h}{\tt posix\+\_\+tests.\+h}
\begin{DoxyItemize}
\item P\+O\+S\+IX user tests
\end{DoxyItemize}
\end{DoxyItemize}
\end{DoxyItemize}

\subsection*{Printf display functions}


\begin{DoxyItemize}
\item \mbox{[}printf\mbox{]}(printf)
\begin{DoxyItemize}
\item Printf and math IO functions
\item \href{printf/mathio.c}{\tt mathio.\+c}
\begin{DoxyItemize}
\item Number conversions
\end{DoxyItemize}
\item \href{printf/mathio.h}{\tt mathio.\+h}
\begin{DoxyItemize}
\item Number conversions
\end{DoxyItemize}
\item \href{printf/n2a.c}{\tt n2a.\+c}
\begin{DoxyItemize}
\item Binary to A\+S\+C\+II converter number size only limited by memory
\end{DoxyItemize}
\item \href{printf/printf.c}{\tt printf.\+c}
\begin{DoxyItemize}
\item My small printf code -\/ with floating point support and user defined low character level IO
\end{DoxyItemize}
\item \href{printf/sscanf.c}{\tt sscanf.\+c}
\begin{DoxyItemize}
\item My small scanf code -\/ work in progress
\end{DoxyItemize}
\item \href{printf/test_printf.c}{\tt test\+\_\+printf.\+c}
\begin{DoxyItemize}
\item Test my printf against glibs 1,000,000 tests per data type
\end{DoxyItemize}
\end{DoxyItemize}
\end{DoxyItemize}

\subsection*{SD Card files for project}


\begin{DoxyItemize}
\item \mbox{[}sdcard\mbox{]}(sdcard)
\begin{DoxyItemize}
\item My H\+P85 A\+M\+I\+GO and S\+S80 disk images
\begin{DoxyItemize}
\item \href{create_images.sh}{\tt create\+\_\+iamges.\+sh}
\begin{DoxyItemize}
\item Linux bash script to build A\+LL the disk images
\begin{DoxyItemize}
\item Files from A\+S\+C\+I\+I-\/files, L\+I\+F-\/files are added to all of the created images
\end{DoxyItemize}
\end{DoxyItemize}
\item \href{sdcard/hpdisk.cfg}{\tt hpdisk.\+cfg}
\begin{DoxyItemize}
\item All Disk definitions, address, P\+PR, D\+E\+B\+UG level for S\+S80 and A\+M\+I\+GO drives
\begin{DoxyItemize}
\item P\+R\+I\+N\+T\+ER address
\end{DoxyItemize}
\end{DoxyItemize}
\item \href{sdcard/amigo.cfg}{\tt amigo.\+cfg}
\begin{DoxyItemize}
\item Alternate configuration for using only A\+M\+I\+GO drives
\begin{DoxyItemize}
\item Use this if your system does not support S\+S80 drives
\begin{DoxyItemize}
\item Copy this file over the hpdisk.\+cfg file after renaming the hpdisk.\+cfg file
\end{DoxyItemize}
\item P\+R\+I\+N\+T\+ER address
\end{DoxyItemize}
\end{DoxyItemize}
\item \href{sdcard/amigo1.lif}{\tt amigo1.\+lif}
\begin{DoxyItemize}
\item A\+M\+I\+GO disk image file number 1
\item Has some demo basic programs in it
\end{DoxyItemize}
\item \href{sdcard/amigo2.lif}{\tt amigo.\+lif}
\begin{DoxyItemize}
\item A\+M\+I\+GO disk image file number 2
\item Has some demo basic programs in it
\end{DoxyItemize}
\item \href{sdcard/ss80-1.lif}{\tt ss80-\/1.\+lif}
\begin{DoxyItemize}
\item S\+S80 hard drive disk image file number 1
\item Has some demo basic programs in it
\end{DoxyItemize}
\item \href{sdcard/ss80-2.lif}{\tt ss80-\/2.\+lif}
\begin{DoxyItemize}
\item S\+S80 hard drive disk image file number 2
\end{DoxyItemize}
\end{DoxyItemize}
\item \href{sdcard/configs}{\tt sdcard/configs}
\begin{DoxyItemize}
\item Copies of the hp85disk config files
\end{DoxyItemize}
\item \href{sdcard/scripts}{\tt sdcard/scripts}
\begin{DoxyItemize}
\item Scripts that help creating L\+IF images from multiple files
\begin{DoxyItemize}
\item Used by \href{create_images.sh}{\tt create\+\_\+iamges.\+sh}
\end{DoxyItemize}
\end{DoxyItemize}
\item \href{sdcard/traces}{\tt sdcard/traces}
\begin{DoxyItemize}
\item My H\+P85 bus trace files
\begin{DoxyItemize}
\item \href{sdcard/traces/amigo_trace.txt}{\tt amigo\+\_\+trace.\+txt}
\begin{DoxyItemize}
\item A\+M\+I\+GO trace file when connected to H\+P85 showing odd out of order command issue
\end{DoxyItemize}
\item \href{sdcard/traces/gpib_reset.txt}{\tt gpib\+\_\+reset.\+txt}
\begin{DoxyItemize}
\item G\+P\+IB reset trace when connected to H\+P85
\end{DoxyItemize}
\item \href{sdcard/traces/gpib_trace.txt}{\tt gpib\+\_\+trace.\+txt}
\begin{DoxyItemize}
\item G\+P\+IB transaction trace when connected to H\+P85
\end{DoxyItemize}
\end{DoxyItemize}
\item \mbox{[}plots\mbox{]}(sdcard/plots\mbox{]}
\begin{DoxyItemize}
\item My H\+P85 plot capture files
\begin{DoxyItemize}
\item \href{sdcard/plots/plot1.plt}{\tt plot1.\+plt}
\item \href{sdcard/plots/plot2.plt}{\tt plot2.\+plt}
\end{DoxyItemize}
\end{DoxyItemize}
\item \href{sdcard/ASCII-files}{\tt A\+S\+C\+I\+I-\/files}
\begin{DoxyItemize}
\item A\+S\+C\+II Basic files -\/ in text format for easy editing
\begin{DoxyItemize}
\item \href{sdcard/ASCII-files/CIRCLE.TXT}{\tt C\+I\+R\+C\+L\+E.\+T\+XT}
\item \href{sdcard/ASCII-files/DRIVES.TXT}{\tt D\+R\+I\+V\+E\+S.\+T\+XT}
\item \href{sdcard/ASCII-files/GPIB-TA.txt}{\tt G\+P\+I\+B-\/\+T\+A.\+txt}
\item \href{sdcard/ASCII-files/HELLO.TXT}{\tt H\+E\+L\+L\+O.\+T\+XT}
\item \href{sdcard/ASCII-files/RWTEST.TXT}{\tt R\+W\+T\+E\+S\+T.\+T\+XT}
\item \href{sdcard/ASCII-files/TREK85A.TXT}{\tt T\+R\+E\+K85\+A.\+T\+XT}
\item \href{sdcard/ASCII-files/TREK85}{\tt A\+S\+C\+I\+I-\/files/\+T\+R\+E\+K85}
\begin{DoxyItemize}
\item T\+R\+E\+K85 by Martin Hepperle, December 2015
\begin{DoxyItemize}
\item \href{https://groups.io/g/hpseries80/topic/star_trek_game_for_hp_85/4845241}{\tt https\+://groups.\+io/g/hpseries80/topic/star\+\_\+trek\+\_\+game\+\_\+for\+\_\+hp\+\_\+85/4845241}
\end{DoxyItemize}
\item \href{sdcard/TREK85/author.txt}{\tt author.\+txt}
\item \href{sdcrad/TREK85/readme.txt}{\tt readme.\+txt}
\item \href{sdcard/TREK85/Start Trek.pdf}{\tt Star Trek.\+pdf}
\item \href{sdcrad/TREK85/TREK85.BAS}{\tt T\+R\+E\+K85.\+B\+AS}
\item \href{sdcard/TREK85/trek.lif}{\tt trek.\+lif}
\end{DoxyItemize}
\end{DoxyItemize}
\end{DoxyItemize}
\item \href{sdcard/LIF-files}{\tt L\+I\+F-\/files}
\begin{DoxyItemize}
\item L\+IF images with a single program in them
\begin{DoxyItemize}
\item Internal names are the same as the L\+IF name without extension
\end{DoxyItemize}
\item \href{sdcard/ASCII-files/GETSAVE.LIF}{\tt G\+E\+T\+S\+A\+V\+E.\+L\+IF}
\item \href{sdcard/ASCII-files/GPIB-T.lif}{\tt G\+P\+I\+B-\/\+T.\+lif}
\item \href{sdcard/ASCII-files/RWTESTB.lif}{\tt R\+W\+T\+E\+S\+T\+B.\+lif}
\item \href{sdcard/ASCII-files/TREK85B.lif}{\tt T\+R\+E\+K85\+B.\+lif}
\end{DoxyItemize}
\item \href{sdcard/LIF-volumes}{\tt L\+I\+F-\/volumes}
\begin{DoxyItemize}
\item L\+IF images with multiple programs in them
\begin{DoxyItemize}
\item \href{sdcard/ASCII-files/85-SS80.LIF}{\tt 85-\/\+S\+S80.\+L\+IF}
\end{DoxyItemize}
\end{DoxyItemize}
\item \href{sdcard/notes}{\tt notes}
\begin{DoxyItemize}
\item G\+E\+T\+S\+AV documenations
\begin{DoxyItemize}
\item G\+E\+T\+S\+A\+VE can be loaded on an H\+P85 to G\+ET and S\+A\+VE Basic text files
\begin{DoxyItemize}
\item N\+O\+TE\+: my lif utilities can translate A\+S\+C\+II files to and from this format
\end{DoxyItemize}
\end{DoxyItemize}
\item Various notes 

 
\end{DoxyItemize}
\end{DoxyItemize}
\end{DoxyItemize}
\end{DoxyItemize}